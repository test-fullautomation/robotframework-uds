% Copyright 2020-2024 Robert Bosch GmbH

% Licensed under the Apache License, Version 2.0 (the "License");
% you may not use this file except in compliance with the License.
% You may obtain a copy of the License at

% http://www.apache.org/licenses/LICENSE-2.0

% Unless required by applicable law or agreed to in writing, software
% distributed under the License is distributed on an "AS IS" BASIS,
% WITHOUT WARRANTIES OR CONDITIONS OF ANY KIND, either express or implied.
% See the License for the specific language governing permissions and
% limitations under the License.

% --------------------------------------------------------------------------------------------------------------

\section{Overview}
The \pkg\ is designed to interface with automotive ECUs using the UDS protocol 
over the DoIP (Diagnostic over IP) transport layer. 
This library abstracts the complexities of UDS communication, allowing users to 
focus on writing high-level test cases that validate specific diagnostic 
services and responses.

\section{UDS Connector (DoIP)}
Currently, the library supports the \rcode{DoIP} (Diagnostic over IP) transport 
layer, which is commonly used in modern vehicles for diagnostic communication. 
DoIP allows for faster data transfer rates and easier integration with 
network-based systems compared to traditional CAN-based diagnostics.

\section{Configuration}
In order to connect and send/receive message properly using the \pkg\, 
certain configurations must be set up:

\begin{itemize}
   \item DoIP Configuration:
         The library requires the IP address and port of the ECU or the gateway 
         through which the ECU is accessed.
   \item Data Identifiers and Codec:
         Define the Data Identifiers (DIDs) and corresponding codecs in the 
         library’s configuration. 
         This enables correct encoding and decoding of data between the test 
         cases and the ECU.
   \item Session Management: 
         Some UDS services may require the ECU to be in a specific diagnostic 
         session (e.g., extended diagnostics). 
         The library should be configured to manage these session transitions 
         seamlessly.
\end{itemize}

\section{Supported UDS Services}
The \pkg\ library supports almost UDS service as defined in 
\href{https://automotive.wiki/index.php/ISO_14229}{ISO 14229}, 
providing comprehensive coverage for ECU diagnostics.

For detailed information on specific services and how to use them, please refer 
to the next section.

\section{Enhancements Usability with ODXTools Integration}
The \pkg\ library comes with 
\href{https://github.com/mercedes-benz/odxtools}{odxtools} fully integrated, 
allowing you to use readable service names instead of dealing with hex IDs.

You can now specify service names directly in your test cases, 
making them more readable and user-friendly.

Example Usage with Readable Service Names

Need to be updated
\begin{robotcode}
*** Settings ***
Library    RobotFramework_UDS

*** Test Cases ***
Read Data By Service Name Test Case
   ReadDataByName    ${ServiceName}
\end{robotcode}

\section{Examples}
To be added
\begin{robotcode}
*** Settings ***
Library    RobotFramework_UDS

*** Test Cases ***
Tester Present Test Case 
   Tester Present

Routine Control Test Case
   Routine Control    ${id}    ${type}

Read Data By Identifier Test Case
   ReadDataByIdentifier    ${did}

Reset ECU Test Case
   Ecu Reset    ${type}
\end{robotcode}